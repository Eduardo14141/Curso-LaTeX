% & Define alineamientos

% Definir el documento
\documentclass[12pt]{report}
%%%%%%%%%%%%%%%%%%%%%%%%%%%%%%%
% Bibliotecas en el documento

\usepackage[spanish]{babel}
\usepackage[top=20mm, bottom=15mm, right=20mm, left=25mm]{geometry}
\usepackage{lipsum}

\usepackage{graphicx}
\usepackage{subcaption}
%%%%%%%%%%%%%%%%%%%%%%%%%%%%%%%

%%%%%%%%%%%%%%%%%%%%%%%%%%%%%%%
% Configuración del documento

\graphicspath{{./figures}}
\setlength{\parindent}{0pt}
\setlength{\parskip}{12pt}
%%%%%%%%%%%%%%%%%%%%%%%%%%%%%%%

%%%%%%%%%%%%%%%%%%%%%%%%%%%%%%%
% Datos del documento

\title{Introducción a \LaTeX}
\date{\today}
\author{Yo mismo}

%%%%%%%%%%%%%%%%%%%%%%%%%%%%%%%

\begin{document}

	\maketitle
	\tableofcontents
	
	%Definir un capítulo nuevo
	\chapter{Mi primer capítulo}
		
		En la figura \ref{fig:latex-logo}
		\lipsum[1][1-11]
		
		\begin{equation}
			\tiny
            \sum_{x=0}^{x=10}{(4x+4)}
        \end{equation}
	
		\begin{figure}[h]
			\centering
			\begin{subfigure}{0.4\textwidth}
				\centering
				\includegraphics[
					width=\linewidth,
					height=40mm,
					keepaspectratio
				]{latex.png}
				\caption{Logo de \LaTeX}
				\label{fig:latex-logo}
			\end{subfigure}
			\hfill
			\begin{subfigure}{0.4\textwidth}
				\centering
				\includegraphics[
					width=\linewidth,
					height=40mm,
					keepaspectratio
				]{computer.png}
				\caption{Imagen de computadora}
				\label{fig:computer-image}
			\end{subfigure}
			\caption{Figura grande}
			\label{fig:grande}
		\end{figure}
		
		\section{Mi primer sección}
			\lipsum[1-2]
			\lipsum[1][1-3]
		
		% Definir una sección nueva
		\section{Mi segunda sección}
			\lipsum[3-5]
			
	\chapter{Empezando con tablas}
		
		En la tabla \ref{table:datos2} vemos que ... \lipsum[1]
		% | -> alt + 124
		\begin{table}[h!]
			\centering
			\begin{subtable}{0.4\textwidth}
				\centering
				\begin{tabular}{| r | c |}
					\hline
					Variables  & Datos \\ \hline
					a & 5 \\ \hline
					a & 5 \\ \hline
					a & 5 \\ \hline
					a & 5 \\ \hline
					a & 5 \\ \hline
				\end{tabular}
				\caption{Experimento 1}
				\label{table:datos1}
			\end{subtable}
			\hfill
			\begin{subtable}{0.4\textwidth}
				\centering
				\begin{tabular}{| r | c |}
					\hline
					Variables  & Datos \\ \hline
					a & 5 \\ \hline
				\end{tabular}
				\caption{Experimento 2}
				\label{table:datos2}
			\end{subtable}
			\caption{Experimentos realizados}
			\label{table:datos}
		\end{table}
		% *{n}c -> Definir n columnas
		\lipsum[1-3]
		
		\begin{longtblr}[
			caption = {Nuestra primer talltblr},
			label = {table:first-talltblr}
		]{
			colspec = {*{5}X},
			hlines,
			vlines,
			width = 1\columnwidth,
			rowhead = 1,
			row{1} = {
				bg = primary,
				fg = white,
				font = \bfseries,
			},
			row{odd[3]} = {
				bg = background
			},
			cell{2-Z}{1} = {
				font = \bfseries
			},
			rows = {
				valign = m,
				halign = c
			}
		}
			Datos & Día 1 & Día 2 & Día 3 & Día 4 \\
			1 & {835\\ 586} & 158 & 878 & 274 \\
	        2 & 285 & 608 & 904 & 285 \\
	        3 & 107 & 230 & 358 & 368 \\
	        4 & 415 & 301 & 182 & 147 \\
	        5 & 520 & 101 & 117 & 103 \\
	        6 & 299 & 370 & 877 & 809 \\
	        7 & 763 & 836 & 888 & 228 \\
	        8 & 783 & 221 & 428 & 637 \\
	        9 & 872 & 597 & 445 & 743 \\
	        10 & 704 & 472 & 948 & 494 \\
	        11 & 418 & 860 & 277 & 533 \\
	        12 & 905 & 845 & 854 & 732 \\
	        13 & 476 & 552 & 466 & 371 \\
	        14 & 278 & 803 & 872 & 265 \\
	        15 & 825 & 430 & 902 & 227 \\
	        16 & 991 & 561 & 810 & 511 \\
	        17 & 733 & 139 & 857 & 640 \\
	        18 & 210 & 448 & 288 & 158 \\
	        19 & 379 & 729 & 768 & 613 \\
	        20 & 951 & 281 & 138 & 150 \\
	        21 & 166 & 786 & 180 & 642 \\
	        22 & 492 & 524 & 504 & 610 \\
	        23 & 899 & 868 & 774 & 827 \\
	        24 & 213 & 750 & 678 & 673 \\
	        25 & 447 & 991 & 340 & 721 \\
	        26 & 447 & 473 & 138 & 391 \\
	        27 & 773 & 584 & 209 & 187 \\
	        28 & 278 & 296 & 239 & 846 \\
	        29 & 445 & 356 & 628 & 425 \\
	        30 & 259 & 486 & 953 & 940 \\
		\end{longtblr}
				
% Termina nuestro documento
\end{document}