\documentclass[12pt]{report}
%%%%%%%%%%%%%%%%%%%%%%%%%%%%%%%
% Bibliotecas en el documento

\usepackage[spanish]{babel}
\usepackage[top=20mm, bottom=15mm, right=20mm, left=25mm]{geometry}
\usepackage{lipsum}

\usepackage{graphicx}
\usepackage{subcaption}
%%%%%%%%%%%%%%%%%%%%%%%%%%%%%%%

%%%%%%%%%%%%%%%%%%%%%%%%%%%%%%%
% Configuración del documento

\graphicspath{{./figures}}
\setlength{\parindent}{0pt}
\setlength{\parskip}{12pt}
%%%%%%%%%%%%%%%%%%%%%%%%%%%%%%%

%%%%%%%%%%%%%%%%%%%%%%%%%%%%%%%
% Datos del documento

\title{Introducción a \LaTeX}
\date{\today}
\author{Yo mismo}

%%%%%%%%%%%%%%%%%%%%%%%%%%%%%%%

\begin{document}
	\maketitle
	\tableofcontents
	\listoffigures
	\listoftables
	
	%\glsfindwidesttoplevelname
	% style
	
	\printglossary[type=\acronymtype, nonumberlist, title={Abreviaciones y siglas}]
	\printglossary[type=\glsdefaulttype, nonumberlist]
	
	%Definir un capítulo nuevo
	\chapter{Empezando con imágenes}
		
		En la \cref{fig:latex-logo} \lipsum[1][1-11]
	
		\begin{figure}[h]
			\centering
			\begin{subfigure}{0.4\textwidth}
				\centering
				\includegraphics[
					width=\linewidth,
					height=40mm,
					keepaspectratio
				]{latex.png}
				\caption{Logo de \LaTeX}
				\label{fig:latex-logo}
			\end{subfigure}
			\hfill
			\begin{subfigure}{0.4\textwidth}
				\centering
				\includegraphics[
					width=\linewidth,
					height=40mm,
					keepaspectratio
				]{computer.png}
				\caption{Imagen de computadora}
				\label{fig:computer-image}
			\end{subfigure}
			\caption{Figura grande}
			\label{fig:grande}
		\end{figure}
		
		\section{Mi primer sección}
			\lipsum[1-2]
			\lipsum[1][1-3] \cite{nadeeshani_nicotinamide_2022}
		
		% Definir una sección nueva
		\section{Mi segunda sección}
			\lipsum[3-5]
			
	\chapter{Empezando con tablas}
		
		En la \cref{table:datos2} vemos que ... \lipsum[1] \cite{cabrera_morales_efectos_2006}
		% | -> alt + 124
		\begin{table}[h!]
			\centering
			\begin{subtable}{0.4\textwidth}
				\centering
				\begin{tabular}{| r | c |}
					\hline
					Variables  & Datos \\ \hline
					a & 5 \\ \hline
					a & 5 \\ \hline
					a & 5 \\ \hline
					a & 5 \\ \hline
					a & 5 \\ \hline
				\end{tabular}
				\caption{Experimento 1}
				\label{table:datos1}
			\end{subtable}
			\hfill
			\begin{subtable}{0.4\textwidth}
				\centering
				\begin{tabular}{| r | c |}
					\hline
					Variables  & Datos \\ \hline
					a & 5 \\ \hline
				\end{tabular}
				\caption{Experimento 2}
				\label{table:datos2}
			\end{subtable}
			\caption{Experimentos realizados}
			\label{table:datos}
		\end{table}
		% *{n}c -> Definir n columnas
		\lipsum[1-3]
		
		\begin{longtblr}[
			caption = {Nuestra primer talltblr},
			label = {table:first-talltblr}
		]{
			colspec = {*{5}X},
			hlines,
			vlines,
			width = 1\columnwidth,
			rowhead = 1,
			row{1} = {
				bg = primary,
				fg = white,
				font = \bfseries,
			},
			row{odd[3]} = {
				bg = background
			},
			cell{2-Z}{1} = {
				font = \bfseries
			},
			rows = {
				valign = m,
				halign = c
			}
		}
			Datos & Día 1 & Día 2 & Día 3 & Día 4 \\
			1 & {835\\ 586} & 158 & 878 & 274 \\
	        2 & 285 & 608 & 904 & 285 \\
	        3 & 107 & 230 & 358 & 368 \\
	        4 & 415 & 301 & 182 & 147 \\
	        5 & 520 & 101 & 117 & 103 \\
	        6 & 299 & 370 & 877 & 809 \\
	        7 & 763 & 836 & 888 & 228 \\
	        8 & 783 & 221 & 428 & 637 \\
	        9 & 872 & 597 & 445 & 743 \\
	        10 & 704 & 472 & 948 & 494 \\
	        11 & 418 & 860 & 277 & 533 \\
	        12 & 905 & 845 & 854 & 732 \\
	        13 & 476 & 552 & 466 & 371 \\
	        14 & 278 & 803 & 872 & 265 \\
	        15 & 825 & 430 & 902 & 227 \\
	        16 & 991 & 561 & 810 & 511 \\
	        17 & 733 & 139 & 857 & 640 \\
	        18 & 210 & 448 & 288 & 158 \\
	        19 & 379 & 729 & 768 & 613 \\
	        20 & 951 & 281 & 138 & 150 \\
	        21 & 166 & 786 & 180 & 642 \\
	        22 & 492 & 524 & 504 & 610 \\
	        23 & 899 & 868 & 774 & 827 \\
	        24 & 213 & 750 & 678 & 673 \\
	        25 & 447 & 991 & 340 & 721 \\
	        26 & 447 & 473 & 138 & 391 \\
	        27 & 773 & 584 & 209 & 187 \\
	        28 & 278 & 296 & 239 & 846 \\
	        29 & 445 & 356 & 628 & 425 \\
	        30 & 259 & 486 & 953 & 940 \\
		\end{longtblr}
	
	\chapter{Empezando con ecuaciones}
		
		\lipsum[1]
		
		
		
		\begin{equation}\label{equ:most-beautiful-eq}
			e^{\pi i} -1 = 0
		\end{equation}
		
		\begin{equation}\label{equ:einstein}
			E = m c^{2}
		\end{equation}
		
		\lipsum[1][1-3]
		
		\begin{align}
			f(x) &= x^{2} + 2xy + y^{2} &
				f(x) &= (x+y)^{2} \\
			g(x) &= x^{2} - 2xy + y^{2} &
				g(x) &= (x-y)^{2}
		\end{align}
		\clearpage
		
		\begin{equation}\label{equ:taylor-series}
			f(x) = \sum_{n=0}^{\infty}{
				\dfrac{f^{(n)}(a)}{n!}(x-a)^{n}
			}
		\end{equation}		
		
		\begin{align} \label{equ:mcclaurin-sinh-series}
			\sinh(x) &= \sum_{n=0}^{\infty}{\dfrac{x^{2n+1}}{(2n+1)!}} &
				\int \sinh(u) du &= \cosh(u) + c
		\end{align}
		
		\begin{align*}
			f &\in L^{1}(\mathbb{R}) & f &\in L^{1}(\mathbb{C})\\
			\hat{f}(\xi) &:= \int_{\infty}^{\infty} f(x) e ^{2\pi\xi x} dx
		\end{align*}
		
		\cref{equ:most-beautiful-eq,equ:taylor-series,equ:mcclaurin-sinh-series}
		
	\chapter{Empezando con listas}
		% enumitem -> dos entornos: itemize, enumerate
		
		\begin{enumerate}[columns=4]
			\item Item 2
			\item Item 3
			\item Item 4
			\item Item 2
			\item Item 3
			\item Item 4
			\item Item 2
			\item Item 3
			\item Item 4
			\item Item 4
		\end{enumerate}
		
		Estos son los ejemplos de citas que mostramos: 
		\begin{itemize}
			\item las múltiples citas se producen con el comando \textbf{\textit{cites}} \cites{nadeeshani_nicotinamide_2022}{kalogirou_solar_2004}
			\item las citas entre paréntesis se producen con \textbf{\textit{parencite}} \parencite{kalogirou_solar_2004}
			\item \textbf{\textit{textcite}} nos produce citas en texto conforme a lo requerido
		\end{itemize}
		
		\section{Glosarios y acrónimos}
			\Gls{fluorescencia}: no tiene que ver con la \gls{energia_termica}, pero usemos un término en plural como: \glspl{colector_solar}.
			
			\begin{itemize}
				\item Imprimos completo \acrfull{iaea}
				\item Imprimir largo \acrlong{onu}
				\item Imprimir corto \acrshort{inin}
			\end{itemize}
		
		\section{Dibujos}
			\lipsum[1-2]
			
			\inputtikz{RectangleAndLine}
			\inputtikz{Diagram}
		
		\section{Diagramas}
			\lipsum[34-36]
			
		\begin{figure}
			\centering
			\inputdiagram{FirstDiagram}
			\caption{Diagrama en forest}
			\label{fig:FirstDiagram}
		\end{figure}
		
			
				
% Termina nuestro documento
	\printbibliography[heading=bibintoc]
\end{document}